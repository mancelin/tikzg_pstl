\documentclass[landscape]{slides}
\usepackage[landscape]{geometry}
\usepackage[dvips]{color}
\usepackage[utf8]{inputenc}
\usepackage[T1]{fontenc}
\usepackage{graphicx}
\usepackage[french]{babel}
\usepackage{listingsutf8}
\usepackage{verbatim}
\usepackage{calc}
\usepackage{tikz}

\newcommand{\spacer}{\rule[-3mm]{0mm}{8mm}}

\title{
	{\large \textbf{PSTL M1 Informatique\\ Création d'une interface graphique pour Tikz}}
	\vspace{13mm}
	\begin{flushright}
{\normalsize Encadrant : Fréderic Pechanski\\}
\end{flushright}
	\vspace{20mm}
\begin{flushright}
	\begin{normalsize}
Ancelin Maxime\\
	Diakhate Aminata\\
\end{normalsize}
\end{flushright}	
}
\date{8 juin 2012}

\begin{document}
\renewcommand{\labelitemi}{$\bullet$}

%%%%%%%%%%%%%%%%%%%%%%%%===Slide 1 - Titre
\maketitle

%%%%%%%%%%%%%%%%%%%%%%%%===Slide 2 - Plan
\begin{slide}
Plan
\begin{itemize}

\item Objectifs du PSTL

\item Réalisation du projet

\item Résultats

\item L'avenir de TikzG

%\item Predefined colors are          % COLORS!!!!
%        \textcolor{blue}{blue}, \textcolor{red}{red}, \textcolor{black}{black},
%        \colorbox{blue}{\textcolor{white}{white}},
 %       \textcolor{cyan}{cyan}, \textcolor{magenta}{magenta},
  %      \textcolor{yellow}{yellow}, \textcolor{green}{green}.
%
 %       Define colors?  Colored formulae?  You bet!
  %        \verb2\definecolor{darkgreen}{rgb}{0,0.4,0}2
   %             \definecolor{darkgreen}{rgb}{0,0.4,0}
    %    \\ \textcolor{darkgreen}{$A = \pi r^2$}:


\end{itemize}
\end{slide}

%%%%%%%%%%%%%%%%%%%%%%%%===Slide 3 - OBJECTIFS DU PSTL
\begin{slide}
Objectifs du PSTL

\begin{itemize}

\item Création d'une interface graphique pour
faciliter\\ l'utilisation de Tikz.

\item Sous-ensemble de Tikz traité : nœuds,
arêtes.

\end{itemize}
\end{slide}




\end{document}