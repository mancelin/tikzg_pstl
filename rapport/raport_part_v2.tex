\documentclass[a4paper]{report}
\usepackage[14pt]{extsizes}
\usepackage[utf8]{inputenc}
%\usepackage[T1]{fontenc}
\usepackage[french]{babel}
\usepackage{makeidx}
\usepackage{graphicx}
\usepackage{listingsutf8}
\usepackage{verbatim}
\usepackage{tikz}
\usepackage{titlesec} 
\title{RAPPORT DU PROJET STL \\ TIKZ}
\author{ANCELIN Maxime & DIAKHATE Aminata}

\lstset{language=Perl}
\lstset{commentstyle=\textit}
\lstset{frame=shadowbox, rulesepcolor=\color{gray}}

\begin{document} 

\tableofcontents
\newpage

\titleformat{\chapter}[hang]{\bf\huge}{\thechapter}{2pc}{} 

\chapter {Gestion du projet}
\section{Difficultés}
\subsection{Installation de perlqt4}
L' installation de perlqt4 nous a pris un peu plus d' une semaine,
car il nous manquait des dépendances.

Au fur et a mesure que nous découvrions ce qu' il fallait ajouter,
nous avons enrichi un script d' installation afin de rendre la reinstallation
plus facile pour les utilisateurs finaux de notre application. Ce
script a été conçu pour fonctionner sous \textit{Ubuntu 12.04} et devrait marcher
sans encombre sous \textit{Debian}.

Sous ces \textit{linux}, il suffit de lancer \textbf{script\_intall\_all.sh}
afin d' installer \textit{perlqt4}, \textit{Scintilla} et tout les packages et modules
\textit{Perl} requis au fonctionnement de l' application.

\subsection{Identification des objets Tikz}
Afin d' identifier précisément chaque objet Tikz, nous attribuons à chaque n{\oe}ud et à chaque arête du graphe une couleur unique, que nous nomerons le \textit{ColorID}.

Cette couleur sera ensuite ajoutée au propriétés de chaque objet, puis on générera une image ou chaque objet sera intégralement colorié de cette couleur. 
Nous nomerons cette image \textit{tmp\_tikz\_IDC.png}

Rappelons que dans le cadre de notre projet nous ne nous intéressons qu'aux n{\oe}uds et arêtes (notés \textbf{node} et \textbf{draw}). L'attribution d'un code couleur a chaque objet identifié nous permet de générer une image temporaire où chaque éléments est intégralement coloré en sa couleur.

\vspace{5mm}
\begin{center}
\begin{tabular}{ccc}
\begin{tikzpicture}
[node distance=40pt]
\node[rectangle,draw] (n1) {a};
\node[circle,double,draw,right of=n1] (n2) {$\sqrt{x}$};
\draw[->] (n1) -- (n2);
\node[below of=n1,right of = n1,node distance=50pt] (n3) {c};
\draw[<->,dashed] (n1) -- (n3);
\end{tikzpicture} &
$\rightarrow$ &
\begin{tikzpicture}[node distance=40pt]
\node[rectangle,draw,red!30!green!30,fill=red!30!green!30] (n1) {a};
\node[circle,double,draw,right of=n1,red!30!green!31,fill=red!30!green!31] (n2) {$\sqrt{x}$};
\draw[line width=5pt,red!30!green!32,fill=red!30!green!32] (n1) -- (n2);
\node[below of=n1,right of = n1,node distance=50pt,red!30!green!33,fill=red!30!green!33] (n3) {c};
\draw[line width=5pt,red!30!green!34,fill=red!30!green!34] (n1) -- (n3);
\end{tikzpicture}
\\ 
tmp\_tikz.png &  & tmp\_tikz\_IDC.png \\ 
\end{tabular} 
\end{center}

Chacuns des objets de tmp\_tikz\_IDC.png a une couleur différente, la variation est juste trés légére.
\\

Tout d'abord, on expliquera comment on peut modifier un code Tikz afin d'attribuer une couleur a un n{\oe}ud ou une arête.

Ensuite, nous introduirons la méthode de génération de couleur unique pour chaque objet et son utilisation pour creer notre image temporaire.

Finalement, nous montrerons comment exploiter l'image colorID dans le cadre de notre projet.

\subsubsection{Comment colorer un objet Tikz intégralement d'une couleur}
Chaque objet Tikz posséde un champ de propriétés. Dans le cas de la couleur et de la couleur de  remplissage (\textit{fill}), la derniére valeur attribuée est la valeur finale.
On peut integralement colorer un objet d'une seule couleur, par exemple en bleu, en ajoutant a la fin des propriétés \textit{blue,fill=blue}.
\begin{center}
\begin{tabular}{ccc}
\begin{tikzpicture}
\node[draw] (n1) {a};
\end{tikzpicture} &
$\rightarrow$ &
\begin{tikzpicture}
\node[draw,blue,fill=blue] (n1) {a};
\end{tikzpicture}
\\ 
\small{\verb?\node[draw] (n1) {a};?}
 &  & \small{\verb?\node[draw,?\textcolor{red}{blue,fill=blue}] (n1) {a};}\\ 
\end{tabular} 
\end{center}

Pour avoir une plus grande variétés de couleurs que celles prédefinies, nous utiliserons le package \textit{xcolor}, qui permet de donnner un pourcentage a une couleur et même de les mélanger. Par exemple : blue!50!red!33
\begin{center}
\begin{tabular}{ccc}
\begin{tikzpicture}
\node[draw] (n1) {a};
\end{tikzpicture} &
$\rightarrow$ &
\begin{tikzpicture}
\node[draw,blue!50!red!33,fill=blue!50!red!33] (n1) {a};
\end{tikzpicture}
\\ 
\small{\verb?\node[draw] (n1) {a};?}
 &  & \small{\verb?\node[draw,?\textcolor{red}{blue!50!red!33,fill=blue!50!red!33}] (n1) {a};}\\ 
\end{tabular} 
\end{center}


\subsubsection{Ajout du colorId aux objets Tikz}
Afin de générer des couleur uniques selon une logique concise, on a mis en place le module perl ColorId.pm. Ce module agit comme une sorte de "compteur" de couleurs et a chaque appel de la fonction \textbf{gen\_next\_ColorId} une chaine du type "\textit{red!y!green!x,fill=red!y!green!x}" est retournée (avec \textit{x} et \textit{y} compris entre 30 et 100).
Cette fonction nous permet de générer jusqu'a 5041 couleurs différentes en incréméntant/reinitialisant deux variables selon ce code :
\vspace{0.8mm}
{\small \lstinputlisting[inputencoding=utf8/latin1]{genNextColorID.pl}}

Il nous reste ensuite a parser chaque élément Tikz et a ajouter aux propriétés de l' objet la chaine retournée par cette fonction.
Nous traiterons le cas des arêtes a part, afin de pouvoir modifier la grosseur du trait pour permettre une meilleure interaction avec les utilisateur.
Ci dessous, une transformation d' un code Tikz aprés l' ajout des ColorID :

\newpage
\begin{small}
\begin{verbatim}
[node distance=40pt]
\node[rectangle,draw] (n1) {a};
\node[circle,double,draw,right of=n1] (n2) {$\sqrt{x}$};
\draw[->] (n1) -- (n2);
\node[below of=n1,right of = n1,node distance=50pt] (n3) {c};
\draw[<->,dashed] (n1) -- (n3);
\end{verbatim}
\begin{flushright}
{\small \textit{code Tikz initial}}
\end{flushright}

\begin{center}
{\normalsize $\downarrow$ \textit{ajout des ColorID}}\\
\end{center}
\verb?[node distance=40pt]?\\
\verb?\node[rectangle,draw?\textcolor{red}{\textbf{,red!30!green!30,fill=red!30!green!30}}\verb?] (n1) {a};?\\
\verb?\node[circle,double,draw,right of=n1?\textcolor{red}{\textbf{,red!30!green!31,fill=red!30!green!31}}\verb?] (n2) {$\sqrt{x}$};?\\
\verb?\draw[?\textcolor{red}{\textbf{line width=5pt,red!30!green!32,fill=red!30!green!32}}\verb?] (n1) -- (n2);?\\
\verb?\node[below of=n1,right of = n1,node distance=50pt?\textcolor{red}{\textbf{,red!30!green!33,fill=red!30!green!33}}\verb?] (n3) {c};?\\
\verb?\draw[?\textcolor{red}{\textbf{line width=5pt,red!30!green!34,fill=red!30!green!34}}\verb?] (n1) -- (n3);?\\
\begin{flushright}
{\small \textit{code Tikz avec ColorID}}
\end{flushright}
\end{small}

\end{document}


