\documentclass[a4paper]{report}
 \usepackage[14pt]{extsizes}
 \usepackage[utf8]{inputenc}
 \usepackage[T1]{fontenc}
 \usepackage[french]{babel}
 \usepackage{makeidx}
 \title{RAPPORT DU PROJET STL \\ TIKZ}
 \author{ANCELIN Maxime & DIAKHATE Aminata}

 \makeindex    %pour générer l'index

\begin{document}          %début du document

 \maketitle
 \tableofcontents
 \newpage

Pour la soutenance résumer. choisir de présenter 
les choses sous formes de diagramme et passer du 
temps sur les difficultés.
2 membres du jury auront lu le rapport. 

	Plan\\
 Introduction\\
 Presentation Générale \\
   Etat Actuel\\
   Motivations\\ 
   Objectifs\\ 
   (Fontionnalites)\\
 Outils\\ 
   Materiels\\ 
   Langage de programmation\\ 
 Developpement\\ 
   Gestion du projet\\
   Implementation\\
   Résultats\\
 Conclusion\\
   (Bilan, Difficultés, ce que le projet nous a apporté)\\ 
 Perspectives

 \newpage
 \section{Introduction}
  Dans le cadre de notre enseignement universitaire à l'UPMC(Université Pierre Marie Curie) en master Informatique, nous devons réaliser un projet. 
  
  \section{Présentation du projet}
  \subsection{Etat Actuel}
  LaTex(Lamport TEX) est un langage et un système de composition de documents. Du fait de sa relative simplicité, il est souvent utilisé dans les domaines techniques et scientifiques pour la production d'un contenu complexe (équations, graphes, ...) ayant une mise en forme standard. Afin d'inclure ces contenus complexes dans les documents en restant dans l'environnement LaTex, LaTex dispose d'un package TikZ permettant d'inclure des figures au format PDF. En effet, TikZ permet d'obtenir des figures géométriques complexes précise et d'une grande qualité. Cependant son apprentissage n'est pas évident. 
  \subsection{Motivation}
  Notre encadrant, qui est lui-même un utilisateur de TikZ, a souhaité faciliter son utilisation afin de permettre à un plus grand nombre d'utilisateurs de profiter des avantages de TikZ sans passer par son apprentissage. Pour ce faire, il nous propose donc de mettre en place une interface graphique permettant la construction  visuelle (et non textuelle) des figures.
  \newline
  Pour ce projet que nous devions réaliser dans le cadre de notre master, nous avions souhaité, avant même de 
connaître les projets proposés, de choisir un projet qui nous permettrait de nous rapprocher le plus des projets qu'on pourrait rencontrer dans le monde professionnel et qui 
  \subsection{Objectifs}

   

\end{document}


